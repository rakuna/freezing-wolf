%Tom McCleery's LaTeX Template: Last Updated 2012-4-14
\documentclass[a4paper,11pt]{article}
\usepackage[T1]{fontenc}
\usepackage[utf8]{inputenc}
\usepackage{lmodern}
\usepackage{graphicx}
\usepackage{subfig}
\usepackage{amssymb}
\usepackage{amsmath}
\usepackage{fancyhdr}

%Changing Default Numbering Style
 %\renewcommand{\thepart}{\Alph{part}}
 %\renewcommand{\thesection}{\Alph{section}}
 %\renewcommand{\thesubsection}{\Alph{subsection}}
 %\renewcommand{\thesubsubsection}{\Alph{subsubsection}}
%Alternatives to {\Alph{}}
 %\alph{}	%lowercase alphabet
 %\Alph{}	%uppercase alphabet
 %\arabic{}	%arabic numerals
 %\roman{}	%roman numerals
%Fancy Alteration Examples
 %\renewcommand{\thesection}{(\Alph{section})}
 %\renewcommand{\thesubsection}{{\thesection}-\arabic{subsection}} 	%A-1
 %
 %\renewcommand{\thesection}{(\Alph{section})}
 %\renewcommand{\thesubsection}{{\thesection}-\roman{subsection}}	%A-i
 %
 %\renewcommand{\thesection}{\roman{section}.}				%i.


%Defining Specific Margins
%Automatic
 %\usepackage{fullpage}
%Manual
 %\addtolength{\oddsidemargin}{-.875in}
 %\addtolength{\evensidemargin}{-.875in}
 %\addtolength{\textwidth}{1.75in}
 %\addtolength{\topmargin}{-.875in}
 %\addtolength{\textheight}{0.85in} %1.75

%\title{ Title of Document } 	%e.g. Laboratory Report 1 \\ ELEC3306: Signals \& Systems 3
%\author{Thomas McCleery \\20368975}
%\date{August 29, 2011}

\begin{document}
%\pagestyle{fancy}
%\lhead{\footnotesize \parbox{11cm}{ELEC3306: Laboratory Report 1 - Thomas McCleery - 20368975} }
%\renewcommand\headheight{24pt}
%\maketitle
%\tableofcontents

%\begin{abstract}
 %Text to go in Abstract
%\end{abstract}


%%%% Document Structure %%%%

%Creating Document Structure
 %\section[ optional: for contents page only ]{ for internal of document }
 %\section{ for contents page and internal of document }
 %\subsection{numbered and displayed on contents page}
 %\subsubsection*{not numbered or displayed on contents page}


%%%% Formatting %%%%

%Useful Formatting Commands
 %\noindent

%Creating Bold Text
 %\textbf{ text goes here }


%%%% Mathematics %%%% 

%Mathematical Notations:
 %\dot{x}
 %
 %\mathcal{C} %%Makes controllability matrix C

%Creating Equations Over Multiple Lines
%
%Method 1:
 %\[
 %\begin{array}{lcl} 
  %	a &=& b\\
  %	c &=& d\\
 %\end{array}
 %\]
%
%Method 2:
 %\begin{eqnarray*}
 %	\dot{x}	&=& ax+bu\\
 % 	y	&=& cx
 %\end{eqnarray*}
%
%Method 3: (This Method Might Number The Equation)
 %\begin{align}
 %    E &= mc^2                              \\
 %    m &= \frac{m_0}{\sqrt{1-\frac{v^2}{c^2}}}
 %\end{align}
%
%Method 4: Multiple line with matrices in each line
%\[\begin{array}{lcc}
%  \bar{A} &=& \left[
%  \begin{array}{cc} \bar{A}_{c} & \bar{A}_{c\bar{c}}\\ 0 & \bar{A}_{\bar{c}}\\ \end{array} \right]\\
%  \bar{b} &=& \left[
%  \begin{array}{c} \bar{b}_{c} \\ 0\\ \end{array} \right]\\
%  \bar{c} &=& \left[
%  \begin{array}{cc} \bar{c}_{c} & \bar{c}_{\bar{c}}\\ \end{array} \right]\\
%\end{array}\]


%%%% Tables %%%%

%Creating Pretty Tables
 %\begin{table}[h!]  %h describes location of float, !overrides normal float positioning
 %  \centering
 %  \begin{tabular}{| l || c | r |}
 %    \hline
 %    a	& b	& c\\
 %    \hline
 %    d	& e	& f\\
 %    g	& h	& i\\
 %    \hline
 %  \end{tabular}
 %  \caption{Description of Table}
 %  \label{tab:label-for-referencing-to-table} %\label MUST be on line below caption for referencing to work correctly
 %\end{table}


%%%% Pictures %%%%
 
 %Inserting Multiple Pictures With A Single Caption
 %\begin{figure}[h]
 %  \centering
 %  \subfloat[Sub Caption]  {\includegraphics[width=0.6\textwidth]{ Image Location.png}}\\	%e.g. Image Location = ./Images/picture.png
 %  \subfloat[Sub Caption]  {\includegraphics[width=0.6\textwidth]{ Image Location.png}}
 %  \caption{Total Caption for pictures}
 %  \label{fig:figure-name-for-referencing}
 %\end{figure}
 

%%%% Internal Referencing %%%%

%Assigning A Label
 %\label{fig:}
 %\label{tab:}

%Citing A Label
 %

%Referencing Tables and Figures
%Gives The Number Assigned To The Given Label
 %~\ref{label-name}
%Gives the Page Number Where The Label Is Located
 %~\pageref{label-name}

%A Nice Way To Display Computer Code In It's Native Format
 %\lstset{ breaklines=true,morecomment=[l]{\%},basicstyle=\footnotesize, numbers=right}
 %\begin{lstlisting}
  %code to be referenced in native format.
 %\end{lstlisting}

\LaTeX \ Template Created By Tom McCleery
\end{document}
